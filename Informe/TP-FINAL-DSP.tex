\documentclass[]{article}
%\documentclass[journal,10pt,draftclsnofoot,onecolumn]{IEEEtran}
%\usepackage{graphics,multirow,amsmath,amssymb,textcomp,subfigure,multirow,xspace,arydshln,cite}

\usepackage[]{graphicx}   % para manejar graficos

\usepackage[space]{grffile} % para manejar graficos

\usepackage{caption}

\usepackage{enumerate}    % para hacer listas numeradas

\usepackage{amsmath}        % no se..

\usepackage{amsfonts}     % no se..

\usepackage{authblk}    % para definir las afiliaciones de cada autor

\usepackage{layout}     % no se..

\usepackage{lineno}

\usepackage[sorting=none]{biblatex}   % para manejar la bibliografia / referencias

\usepackage{lipsum}     % para generar texto random

\usepackage{multicol}   % para usar dos columnas

\usepackage{palatino}   % para que la fuente sea palatino

\usepackage[utf8]{inputenc} % para poder usar tildes

\usepackage[spanish]{babel} % para escribir en español

\addto\captionsspanish{\def\tablename{Tabla}} % cambiar ``cuadro'' por ``Tabla''

\usepackage[sc,big,raggedright,bf]{titlesec} % para definir el formato del header de cada seccion.

\usepackage[font=small]{caption} % para que la fuente de un epigrafe no tenga el mismo tamaño que el cuerpo del texto

\usepackage{geometry}
 \geometry{
 a4paper,
 textwidth={17cm},
 textheight={23cm},
 left={2cm},
 top={2.5cm},
 }

\setlength{\columnsep}{1cm} % para que la separacion entre columnas sea de 1 cm

\graphicspath {{imagenes/}}

\defbibheading{bibliography}{\section{\refname}} % para que bibtex no imponga su header cuando uso \printbibliography, y que se use el de babel

\addbibresource{bibliografia.bib} % para importar el archivo .bib

\title{\textbf{\LARGE{\textsf{IMPLEMENTACIÓN DE UN REVERBERADOR DIGITAL DE SCHROEDER}}}}
 % defino el titulo del Paper

\date{} % lo pongo vacio para que no aparezca abajo del abstract

\newcommand{\figura}[3]{
\begin{figurehere}
\centering
\includegraphics[width=\linewidth]{#1}
\captionof{figure}{#2}
\label{#3}
\end{figurehere}
}
\newcommand{\tabla}[4]{
\begin{tablehere}
\begin{center}
\begin{tabular}{#1}
#2
\end{tabular}
\caption{#3}
\label{#4}
\end{center}
\end{tablehere}
}

\usepackage{fancyhdr}

\usepackage{hyperref}
%%%%%%%%%%%%%%%%%%%%%%%%%%%%%%%%%%%%%%%%%%%%%%%%%%%%%%%%%%%%%%%%%%%%%%%%%%%%%%%%%
% http://www-h.eng.cam.ac.uk/help/tpl/textprocessing/multicol_hint.html
\makeatletter           % esto lo uso para poder definir figuras
\newenvironment{tablehere}    % esto lo uso para poder definir figuras
  {\def\@captype{table}}    % esto lo uso para poder definir figuras

  {}              % esto lo uso para poder definir figuras
                  % esto lo uso para poder definir figuras
\newenvironment{figurehere}   % esto lo uso para poder definir figuras
  {\def\@captype{figure}}   % esto lo uso para poder definir figuras
  {\par\medskip}
  {}              % esto lo uso para poder definir figuras
\makeatother          % esto lo uso para poder definir figuras
%%%%%%%%%%%%%%%%%%%%%%%%%%%%%%%%%%%%%%%%%%%%%%%%%%%%%%%%%%%%%%%%%%%%%%%%%%%%%%%%%

%%%%%%%%%%%%%%%%%%%%%%%%%%%%%%%%%%%%%%%%%%%%%%%%%%%%%%%%%%%%%%%%%%%%%%%%%%%%%%%%%
%               ACA EMPIEZA EL DOCUMENTO                            %
%%%%%%%%%%%%%%%%%%%%%%%%%%%%%%%%%%%%%%%%%%%%%%%%%%%%%%%%%%%%%%%%%%%%%%%%%%%%%%%%%


\begin{document} % empieza el documentoo


\renewcommand{\headrulewidth}{0pt} % para que no haya linea decorativa en el header.


\author[1]{Federico Feldsberg} % defino el autor
\affil[1]{Universidad Nacional de Tres De Febrero, Buenos Aires, Argentina \newline \texttt{fedefelds@hotmail.com}} % afiliacion del autor


\begin{minipage}[h]{\textwidth} % uso el entorno minipage para que el abstract este en la misma pagina que el titulo
    \maketitle
    \thispagestyle{fancy}
    \fancyhf{}
    \rhead{\today}
    \lhead{Procesamiento digital de señales}
    \cfoot{\thepage}

\end{minipage}


\begin{abstract}

\textit{\lipsum[1]}
\end{abstract}

\vspace{0.8 cm}% Additional space between abstract & rest of document

\begin{multicols}{2}
\section{Marco teórico}

En los años 60, Schroeder y Logan propusieron un arreglo
capaz de generar la respuesta ``Natural'' de una sala reverberante
\cite{schroeder1961,schroeder1962}. El término ``natural'' implica una falta de coloración espectral y una alta
concentración de ecos por segundo. Schroeder y Logan señalan que los métodos de
reverberación disponibles en su momento carecian de dicha ``naturalidad''.

Este desarrollo resulto ser sumamente valioso debido a que el mismo fue capaz de
suplir ambas falencias de las técnicas de reverberación disponibles en su momento.

En una primer aproximación, Schroeder y Logan proponen una linea de retardo realimentada,
ilustrada en la figura \ref{fig:mk2}:

\figura
{mk2}
{Linea de retardo realimentada \cite{Reiss}.}
{fig:mk2}
La respuesta al impulso de dicho sistema esta dada por:
\begin{equation}
  h(t)=\sum_{n=0}^{\infty}\:g^n \:
  \delta(t-n\:\tau)\quad \text{con} \quad |g|<1
  \label{eq:mk2}
\end{equation}

En el dominio temporal, la ecuación \ref{eq:mk2} se asemeja a una cantidad infinita
de impulsos, desplazados y escalados por un factor que decrece exponencialmente. Es por eso
que parece ser un resultado valioso. Sin embargo, en el dominio de la frecuencia,
este primer sistema posee un alto grado de coloración. Esto se debe a que el mismo
se asemeja a un filtro peine y es por ello que dicho sistema no es un candidato de
reverberador ``natural''.

El gran avance de Scrhoeder y Logan consiste en haber descubierto que al hacer ciertas
modificaciones al sistema presentado en la figura \ref{eq:mk2}, es posible
lograr una respuesta plana y por lo tanto carente de coloración.
Para lograr dicha respuesta en frecuencia, se combinan la señal sin procesar y
la señal procesada mediante cierto criterio establecido por Schroeder y Logan.

Con estas modificaciones, se obtiene el sistema ilustrado en la figura \ref{fig:mk3}:
\figura
{mk3}
{Unidad básica de reverberación \cite{Reiss}.}
{fig:mk3}

La respuesta en frecuencia de dicho sistema resulta ser plana, pero en si misma
no es capaz de generar una gran densidad de eco. Schroeder y Logan recomiendan
un valor minimo de 1000 ecos por segundo. Podemos decir que el sistema presentado
en la figura \ref{fig:mk3} constituye una unidad básica de reverberación, cuya
respuesta en frecuencia es plana.

De ahora en adelante, nos referiremos a toda unidad básica de reverberación como
filtro pasa-todo.

Para solucionar la escasa densidad de ecos, Schroeder y Logan proponen el siguiente arreglo:

\figura
{mk4}
{Diagrama en bloques del sistema \cite{schroeder1962}. HAY QUE CAMBIAR ESTA FIGURAAA}
{fig:mk4}

El mismo esta compuesto por tres secciones: La primer sección consta de un banco
de filtros peine en paralelo, tales como los de la figura \ref{fig:mk2}, por el
cual ingresa la señal a procesar. Todos los elementos del sistema pueden ser caracterizados
en función de su ganancia de realimentación $g_i$ y su retardo $\tau_i$ con
$1 \leq i \leq 4$. Los valores de dichos parametros estan dados por:

\begin{equation}
  T=\frac{3 \: \tau_i}{-\log|g_i|}
\end{equation}
y
\begin{equation}
  \tau_i \quad \text{Entre 30 y 45 mseg}
\end{equation}
Donde $T$ es el tiempo de reverberación.
Esta primer sección no tiene una respuesta en frecuencia plana, sino que al ser una
conexión en paralelo de varios filtros peine, es esperable que sea bastante irregular \cite[446]{lathi}.
Este desvío de la respuesta plana esta justificado por experimentos desarrollados
en los laboratorios de Bell. Estos experimentos indicarían que el oido humano no es capaz de
distinguir entre una sala con respuesta plana y una sala con una gran cantidad
de irregularidades en su respuesta en frecuencia.

La segunda sección consta de dos de filtros pasa-todo conectados en serie cuyos valores de
$\tau_i$ oscilan entre 5 y 1.7 mseg. A su vez, se sugiere que $g_{5}$ y $g_{6}$ sean
iguales a 0,7.

La tercera y última sección consiste de una etapa de ganancia $g_7$ la cual permite
determinar la mezcla entre señal reverberada y no reverberada.


\section{Implementación}
Para la implementación se recurrió al entorno de programación Python, el cual
permite una rápida implementación de filtros FIR e IIR mediante el método \emph{lfilter}, disponible
en la libreria \emph{Scipy}.

Dado un sistema S cuya función de transferencia sea de la forma:
\begin{equation}
  H_S(z)=\frac{\sum_{i=0}^{D} a_i \: z^{-i} }{\sum_{i=0}^{D} b_i \: z^{-i}}
\end{equation}
Dicho sistema puede ser implementado al especificar los coeficientes $a_i$ y $b_i$.

Por lo tanto, se implementaron los filtros peine y los filtros pasa-todo mediante
dos funciones diferentes. La primer función recibe como argumentos de entrada
la cantidad de retardo, la ganancia de realimentación y la señal a
filtrar. El mismo script devuelve la señal filtrada por el filtro correspondiente.

En primer lugar, se siguieron las sugerencias de Reiss \cite[259]{Reiss}.
El diagrama en bloques de la implementación presentada es muy similar al de la
figura \ref{fig:mk4}:

\figura
{mk5}
{Implementación presentada en \cite{Reiss}}
{fig:mk5}


La única diferencia entre los diagramas de las figuras \ref{fig:mk5} y \ref{fig:mk4}
es que este último no contempla la tercer etapa, cuyo fin es permitir mezclar la
señal reverberada y la señal no reverberada.

En el dominio $z$, un  filtro pasa-todo como el de la figura \ref{fig:mk3} esta dado
por \begin{equation}
  AP_{d}^g=\frac{z^{-d}-g}{1-g\:z^{-d}}
  \label{eq:ap_reiss}
\end{equation}

mientras que un filtro peine como el de la figura \ref{fig:mk2} esta dado por:
\begin{equation}
  FBCF_{d}^g=\frac{z^{-d}}{1-g\:z^{-d}}
  \label{eq:fbcf_reiss}
\end{equation}

Se puede ver que las ecuaciones \eqref{eq:fbcf_reiss} y \eqref{eq:ap_reiss} estan expresadas
en función de la ganancia de realimentación $g$ y cantidad de muestras $d$.
Por otro lado, al hablar de retardos Schroeder siempre habla en terminos de
milisegundos. Es por eso que al trabajar en el dominio digital, se debe convertir  segundos a muestras mediante
la siguente ecuación:
\begin{equation}
d_i=\tau_i \cdot F_s
\end{equation}
Donde el numero de muestras equivalentes a un retardo de $\tau_i$ segundos
es función de la frecuencia de muestreo $F_s$

Cabe destacar que la ecuación \ref{eq:fbcf_reiss} no es la única implementación
posible de un filtro peine: En \cite{smith}, Smith propone la siguiente función
de transferencia:

\begin{equation}
FBCF_{d}^g=\frac{1}{1-g\:z^{-d}}
\end{equation}

En el caso de la implementación del filtro pasa-todo la Tanto Reiss como Smith
proponen la misma función de transferencia, dada por la ecuación
\eqref{eq:ap_reiss}.
% Para la ecuación 5, los valores de $d_i$ sugeridos son números primos
% The delay line lengths di are typically mutually prime and spanning suc- cessive orders of magnitude. The 100 ms value was chosen so that when g = 0.708 in Equation (11.10), the time to decay 60 dB (T60) would be 2 s. Thus, for i = 0, T60 ~ 2, and each successive allpass has an impulse response duration that is about a third of the previous one.
\section{Análisis de los resultados}


%  Criticas: no es una caida exponenial, falta de difusion (sch.1962)
\printbibliography
\end{multicols}
\end{document}
