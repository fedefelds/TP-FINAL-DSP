\documentclass[]{article}
%\documentclass[journal,10pt,draftclsnofoot,onecolumn]{IEEEtran}
%\usepackage{graphics,multirow,amsmath,amssymb,textcomp,subfigure,multirow,xspace,arydshln,cite}

\usepackage[]{graphicx}   % para manejar graficos

\usepackage[space]{grffile} % para manejar graficos

\usepackage{caption}

\usepackage{enumerate}    % para hacer listas numeradas

\usepackage{amsmath}        % no se..

\usepackage{amsfonts}     % no se..

\usepackage{authblk}    % para definir las afiliaciones de cada autor

\usepackage{layout}     % no se..

\usepackage{lineno}

\usepackage[sorting=none]{biblatex}   % para manejar la bibliografia / referencias

\usepackage{lipsum}     % para generar texto random

\usepackage{multicol}   % para usar dos columnas

\usepackage{palatino}   % para que la fuente sea palatino

\usepackage[utf8]{inputenc} % para poder usar tildes

\usepackage[spanish]{babel} % para escribir en español

\addto\captionsspanish{\def\tablename{Tabla}} % cambiar ``cuadro'' por ``Tabla''

\usepackage[sc,big,raggedright,bf]{titlesec} % para definir el formato del header de cada seccion.

\usepackage[font=small]{caption} % para que la fuente de un epigrafe no tenga el mismo tamaño que el cuerpo del texto

\usepackage{geometry}
 \geometry{
 a4paper,
 textwidth={17cm},
 textheight={23cm},
 left={2cm},
 top={2.5cm},
 }

\setlength{\columnsep}{1cm} % para que la separacion entre columnas sea de 1 cm

\graphicspath {{imagenes/}}

\defbibheading{bibliography}{\section{\refname}} % para que bibtex no imponga su header cuando uso \printbibliography, y que se use el de babel

\addbibresource{bibliografia.bib} % para importar el archivo .bib

\title{\textbf{\LARGE{\textsf{IMPLEMENTACIÓN DE UN REVERBERADOR DIGITAL DE SCHROEDER}}}}
 % defino el titulo del Paper

\date{} % lo pongo vacio para que no aparezca abajo del abstract

\newcommand{\figura}[3]{
\begin{figurehere}
\centering
\includegraphics[width=\linewidth]{#1}
\captionof{figure}{#2}
\label{#3}
\end{figurehere}
}
\newcommand{\tabla}[4]{
\begin{tablehere}
\begin{center}
\begin{tabular}{#1}
#2
\end{tabular}
\caption{#3}
\label{#4}
\end{center}
\end{tablehere}
}

\usepackage{fancyhdr}

\usepackage{hyperref}
%%%%%%%%%%%%%%%%%%%%%%%%%%%%%%%%%%%%%%%%%%%%%%%%%%%%%%%%%%%%%%%%%%%%%%%%%%%%%%%%%
% http://www-h.eng.cam.ac.uk/help/tpl/textprocessing/multicol_hint.html
\makeatletter           % esto lo uso para poder definir figuras
\newenvironment{tablehere}    % esto lo uso para poder definir figuras
  {\def\@captype{table}}    % esto lo uso para poder definir figuras

  {}              % esto lo uso para poder definir figuras
                  % esto lo uso para poder definir figuras
\newenvironment{figurehere}   % esto lo uso para poder definir figuras
  {\def\@captype{figure}}   % esto lo uso para poder definir figuras
  {\par\medskip}
  {}              % esto lo uso para poder definir figuras
\makeatother          % esto lo uso para poder definir figuras
%%%%%%%%%%%%%%%%%%%%%%%%%%%%%%%%%%%%%%%%%%%%%%%%%%%%%%%%%%%%%%%%%%%%%%%%%%%%%%%%%

%%%%%%%%%%%%%%%%%%%%%%%%%%%%%%%%%%%%%%%%%%%%%%%%%%%%%%%%%%%%%%%%%%%%%%%%%%%%%%%%%
%               ACA EMPIEZA EL DOCUMENTO                            %
%%%%%%%%%%%%%%%%%%%%%%%%%%%%%%%%%%%%%%%%%%%%%%%%%%%%%%%%%%%%%%%%%%%%%%%%%%%%%%%%%


\begin{document} % empieza el documentoo


\renewcommand{\headrulewidth}{0pt} % para que no haya linea decorativa en el header.


\author[1]{Federico Feldsberg} % defino el autor
\affil[1]{Universidad Nacional de Tres De Febrero, Buenos Aires, Argentina \newline \texttt{fedefelds@hotmail.com}} % afiliacion del autor


\begin{minipage}[h]{\textwidth} % uso el entorno minipage para que el abstract este en la misma pagina que el titulo
    \maketitle
    \thispagestyle{fancy}
    \fancyhf{}
    \rhead{\today}
    \lhead{Procesamiento digital de señales}
    \cfoot{\thepage}

\end{minipage}


\begin{abstract}

\textit{\lipsum[1]}
\end{abstract}

\vspace{0.8 cm}% Additional space between abstract & rest of document

\begin{multicols}{2}
\section{Marco teórico}

En los años 60, Schroeder y Logan propusieron un arreglo
capaz de generar la respuesta ``Natural'' de una sala reverberante
\cite{schroeder1961,schroeder1962}. El término ``natural'' implica una falta de coloración espectral y una alta
concentración de ecos por segundo. Schroeder y Logan señalan que los métodos de
reverberación disponibles en su momento carecian de dicha no ``naturalidad''.

Este desarrollo resulto ser sumamente valioso debido a que el mismo fue capaz de
suplir ambas falencias de las técnicas de reverberación disponibles en su momento.

En una primer aproximación, Schroeder y Logan proponen una linea de retardo realimentada,
ilustrada en la figura \ref{fig:mk2}:

\figura
{prueba}
{Linea de retardo realimentada}
{fig:mk2}
La respuesta al impulso de dicho sistema esta dada por:
\begin{equation}
  h(t)=\sum_{n=0}^{\infty}\:g^n \:
  \delta(t-n\:\tau)\quad \text{con} \quad |g|<1
  \label{eq:mk2}
\end{equation}

En el dominio temporal, la ecuación \ref{eq:mk2} se asemeja a una cantidad infinita
de impulsos, desplazados y escalados por un factor que decrece exponencialmente. Es por eso
que parece ser un resultado valioso. Sin embargo, en el dominio de la frecuencia,
este primer sistema posee un alto grado de coloración. Esto se debe a que el mismo
se asemeja a un filtro peine y es por ello que dicho sistema no es un candidato de
reverberador ``natural''.

El gran avance de Scrhoeder consiste en haber descubierto que al hacer ciertas
modificaciones al sistema presentado en la figura \ref{eq:mk2}, es posible
lograr una respuesta plana y por lo tanto carente de color.
Para lograr dicha respuesta en frecuencia, se combinan la señal sin procesar y
la señal procesada mediante cierto criterio establecido por Schroeder y Logan.

Con estas modificaciones, se obtiene el sistema ilustrado en la figura \ref{fig:mk3}:
\figura
{prueba}
{Unidad básica de reverberación}
{fig:mk3}

La respuesta en frecuencia de dicho sistema resulta ser plana, pero en si misma
no es capaz de generar una gran densidad de eco. Schroeder y Logan recomiendan
un valor minimo de 1000 ecos por segundo. Podemos decir que el sistema presentado
en la figura \ref{fig:mk3} constituye una unidad básica de reverberación.

Para solucionar esto ultimo, Schroeder y Logan proponen el siguiente arreglo:

\figura
{mk4}
{Diagrama en bloques del sistema \cite{schroeder1962}.}
{fig:mk4}

El mismo presenta un arreglo en paralelo de filtros peine y dos unidades básicas
de reverberación en serie.

To insure a sufficient echo density several all-pass rever-
berators of the kind shown in Fig. 2 have to be connected in series with the comb filters. The number of echoes per
'second of a comb filter having a delay of 0.04 sec is 25. Four such comb filters in parallel give 100 echoes per second,
which is short by a factor of 10 from the goal of 1,000 echoes per second. Thus, two all-pass reverberator units, each
multiplying the number of echoes by about 3, are requ
Según Reiss,\cite[pp. 10--15]{Reiss}

\section{Implementaciónes}
\lipsum[2]
\printbibliography
\end{multicols}
\end{document}
